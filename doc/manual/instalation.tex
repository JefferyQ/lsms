\chapter{Compiling and running \LSMS}
\section{Requirements}
\LSMS\ requires a Unix like environment (\textit{e.g.}
Linux or MacOS~X) with C++ and Fortran compilers and the following libraries:
\begin{itemize}
\item BLAS linear algebra library
\item LAPACK linear solvers
\item MPI message passing library for parallelization
\item CUDA optionally for GPU acceleration, the code can be built without it.
\end{itemize}

\section{Compilation}
Building \LSMS\ uses \texttt{make} with the main \texttt{Makefile} in the top level directory of the distribution. This file includes a system specific file, \texttt{architecture.h}, that describes the specific environment. Examples for this file are provided in \texttt{architecture}.


\section{Running}

The main input to describe the calculation and the system is provided by an
inputfile. The default name of the input file is \texttt{i\_lsms}.
The input file uses the \texttt{lua} scripting languange (\texttt{www.lua.org}) whic allows
powefull pre-processing of the input.
\LSMS\ will read specially named variables in the \texttt{lua} input that are listed in the chapter describing
the \LSMS\ input.

\subsection{\texttt{lsms}}

\subsection{\texttt{wl-lsms}}
