\chapter{Input File Format}
The main input file format of \LSMS\ employes the \texttt{lua} language.
This allows the processing of the input file before it is read by the
\LSMS\ code itself. The input parameters are read from the values of
gloabl variables that are listed in the following sections.
These variable names should be avoided for any other uses in the \texttt{lua}
input script to avoid unexpected behavior.


\section{Crystal Structure}

All units for atom position and lattice constants are Bohr radii $a_0=0.529177\mathrm{\AA}$.
\subsection{\texttt{bravais}}
(no default value)

\subsection{\texttt{num\_atoms}}
The number of atoms in the unit cell. (no default value) This has to
be the number of sites described by the \texttt{site} entries in the
input file.

%%%%%%%
\subsection{\texttt{site}}

\subsubsection{\texttt{site.pos}}

\subsubsection{\texttt{site.evec}}

\subsubsection{\texttt{site.atom}}

\subsubsection{\texttt{site.pot\_in\_idx}}

\subsubsection{\texttt{site.Z}}

\subsubsection{\texttt{site.Zc}}

\subsubsection{\texttt{site.Zs}}

\subsubsection{\texttt{site.Zv}}

\subsubsection{\texttt{site.lmax}}

\subsubsection{\texttt{site.rLIZ}}

\subsubsection{\texttt{site.rstep}}

\subsubsection{\texttt{site.rad}}

%%%%%%
\subsection{\texttt{xRepeat, yRepeat, zRepeat}}
These shoud be integers greater or equal to one (default value: 1)
that specify the repeat of the unit cell described by
\texttt{bravais}, \texttt{num\_atoms} and \texttt{site} in the
\textit{x, y} and \textit{z} directions specified by \texttt{bravais}.
The total number of atoms in the system will be $\texttt{bravais}
\times \texttt{xRepeat} \times \texttt{yRepeat} \times \texttt{zRepeat}$.

\subsection{\texttt{makeTypesUnique}}
Possible values: 0 or 1 (default value: 1). If set to one all sites are forced to have
independent potentials, thus ignoring equivqlent atom potentials
specified in the input file or in conjunction with
\texttt{\{x,y,z\}Repeat} to generate a repeated unit cell (usefull for
Monte-Carlo simulations).

%%%%%%%%%
%%%%%%%%%
%%%%%%%%%
\section{Control of the Calculation}

\subsection{\texttt{systemid}}
A short string to identify the system and used to construct the
default filenames for the input and output potential files.



\subsection{\texttt{system\_title}}

\subsection{\texttt{potential\_file\_in}}
default: $\texttt{v\_}\langle\textit{systemid}\rangle$

\subsection{\texttt{potential\_file\_out}}
default: $\texttt{w\_}\langle\textit{systemid}\rangle$

\subsection{\texttt{pot\_in\_type}}
The type of the input potential file. The possible values are 0 for
the LSMS~1 HDF5 file format and 1 for the SSITE text format.
(default value: 0, i.e. HDF5)

\subsection{\texttt{pot\_out\_type}}
The file format for the output potential. The file formats are the
same as for \texttt{pot\_in\_type} with the addition of $-1$ to
suppress the generation of an output potential.
(default value: $-1$, no output potential)

\subsection{\texttt{iprint}}

\subsection{\texttt{default\_iprint}}

\subsection{\texttt{print\_node}}

\subsection{\texttt{istop}}
default: \texttt{main}

\subsection{infoEvecFileOut}
default: \texttt{info\_evec\_out}

\subsection{infoEvecFileIn}
default: don't read evecs from a file.

\subsection{\texttt{gpu\_threads}}

%%%%%
%%%%%
\subsection{\texttt{nspin}}
This parameter selects the treatment of magnetism. Possible values are: 1 for non spin-polarized calculations,
2 for spin-polarized collinear clacluations and 3 for non-collinear spin polarized calculations.
The default value fon non relativistic and scalar relativistic
calculations is 3. In the fully relativistic
calculations this parameter is ignored and assumed to be 3 (always non-collinear!). Note that this is different from LSMS\_1
where a fully relativistic calculation was requested by \texttt{nspin=4}, now this is selected by \texttt{relativity="full"}.

\subsection{\texttt{mtasa}}
This parameter selects between muffin-tin potential or atomix sphere
approximation treatment of the atomic potentials. Default value: 0 for
muffin-tin. \textit{This parameter will be replaced by the}
\texttt{potential} \textit{parameter in a future version.}

\subsection{\texttt{fixRMT}}
default: 0

\subsection{\texttt{potential} not yet implemented}
Possible values: \texttt{spherical}, \texttt{muffin-tin},
\texttt{atomic-sphere-approximation} or \texttt{full-potential}

\subsection{\texttt{xcFunctional}}
default: \{0, 1\}, Built in von Barth-Hedin

\subsection{\texttt{relativity}}
This parameter selects the treatment of the valence electrons as non-relativistic, scalar-relativistic
or full-relativistic (\textit{i.e.} the Dirac equation is solved for the valence electrons.)

Possible values: \texttt{none}, \texttt{scalar} or \texttt{full}.

Default value: \texttt{scalar} 

\subsection{\texttt{core\_relativity}}
Treatment of the core electrons either as non-relativistic (Schr\"odinger equation)
or fully relativistic (Dirac equation).
The \texttt{default} is to solve the Schr\"odinger equation for the core electrons if the valence
electrons are non-relativistic and the Dirac equation in the scalar and fully relativistic cases.
 
Possible values: \texttt{default}, \texttt{none} or \texttt{full}

Default value: \texttt{default}

\subsection{\texttt{nscf}}
The maximum number of selfconsistency steps tp be performed.

\subsection{\texttt{writeSteps}}
The frequency of writing potentials during a selfconsistent
calculation. The output potential is written every \texttt{writeSteps}
iteration steps. Note the potential always is written at the end of
the calculation (if the output type is not negative.)
default: 30000 (for practical purposes, the potential never gets
written with the default!)

\subsection{\texttt{rmsTolerance}}
The convergence criterion for selfconsistent iterations. The convergence criterion is reached when
the charge RMS difference is smaller than \texttt{rmsTolerance} for every site.

Default values: $10^{-8}$

\subsection{\texttt{zblockLUSize}}
default: 0, LSMS tries to find the optimal size for a given architecture.

%%%%%
\subsection{\texttt{energyContour}}

\subsubsection{\texttt{energyContour.grid}}

\subsubsection{\texttt{energyContour.npts}}

\subsubsection{\texttt{energyContour.ebot}}

\subsubsection{\texttt{energyContour.etop}}

\subsubsection{\texttt{energyContour.eitop}}

\subsubsection{\texttt{energyContour.eibot}}

\subsubsection{\texttt{energyContour.maxGroupSize}}

%%%%
\subsection{\texttt{numberOfMixingQuantities}}

\subsection{\texttt{mixing}}

\subsubsection{\texttt{mixing.quantity}}

\subsubsection{\texttt{mixing.algorithm}}
possible values: \texttt{simple}, \texttt{broyden}

\subsubsection{\texttt{mixing.mixing\_parameter}}

%%%% Wang Landau for alloys

\subsection{\texttt{alloy\_file\_in}}
default: $\texttt{bank\_v\_}\langle\textit{systemid}\rangle$

\subsection{\texttt{alloy\_file\_out}}
default: $\texttt{bank\_w\_}\langle\textit{systemid}\rangle$

\subsection{\texttt{nalloy\_class}}
number of site occupation classes.
default: 0

\subsection{\texttt{alloy\_class}}
\subsubsection{\texttt{alloy\_class.ncomp}}
number of possible elements in this alloy class
\subsubsection{\texttt{alloy\_class.comp.atom}}
\subsubsection{\texttt{alloy\_class.comp.conc}}
Concentration
\subsubsection{\texttt{alloy\_class.comp.Z}}
\subsubsection{\texttt{alloy\_class.comp.Zc}}
\subsubsection{\texttt{alloy\_class.comp.Zs}}
\subsubsection{\texttt{alloy\_class.comp.Zv}}


